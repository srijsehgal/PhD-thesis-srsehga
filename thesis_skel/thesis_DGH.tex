% !TEX root = mythesis.tex

%==============================================================================
\chapter{Detector exposure and Limits to the diffuse flux}
\label{sec:align}
%==============================================================================
 
The chapter aims to detail the procedure used to calculate the detector exposure for neutrinos for DGL region. The efficiencies calculated in the last chapter are used and final expected neutrino rates are calculated. The first part of the chapter describes the exposure calculation along with the exposure contribution from different channels. Systematic uncertainties which can arise during the full analysis along with their contribution to the exposure are also discussed. 
The second part of the chapter details the results of the unblinding where no possible neutrino candidates were found for the angular range. Using this information an upper limit on the incoming flux of UHE $\nu$ is calculated. This limit is further compared to the one obtained by the previous analysis for the same time period but without the contribution of new triggers. The overall improvement forms a crucial result of this work. 


\section{Exposure Calculation}
\label{sec:det_exposure_calc}

One of the most accurate techniques used at Auger to calculate the exposure of the SD array to UHE$\nu$ is through extensive simulations of different detector configurations. In this method MC neutrino showers are thrown over varied detector configuration to calculate the effective or active area at each instant of time. Since the detector configuration of the SD array is constantly changing(\textit{faulty tanks, regular maintenance etc.}), sometimes on a daily basis, this technique requires a large amount of CPU time and resources making it less desirable for use in this analysis. 

In this analysis a different approach based on the 6T5 condition which is required for each selected event in this analysis is used for the exposure calculation. For this calculation, the 6T5 hexagon is taken to be the smallest possible detection unit for the $\nu$ event. The effective area i.e. the area seen by the incident cosmic neutrino for this detection unit at full efficiency is given by the Brillouin area~\cite{}, $A_{6T5} = 1.95km^2$ as shown by the shaded area in fig~\ref{}. The aperture for this detector unit is dependent on the energy of the primary neutrino ,the slant depth in the atmosphere, X, neutrino flavor, type of the interaction, zenith, azimuth and the point of impact of the shower on the ground. The effective \textit{acceptance} of the detector unit can be written as follows:
\begin{equation}
  A_{hex}(E_{\nu}, X)  = \int^{\phi = 2\pi}_{\phi = 0} \,d\phi \int_{\theta_{min} = 58.5^{\circ}}^{\theta_{max}= 76.5^{\circ}}  \, A_{6T5} \cdot \varepsilon(E_{\nu}, X, \theta) \cdot sin\theta cos\theta \cdot d\theta    \mathrm{[cm^2 sr]}
\end{equation}

,where $\varepsilon(E_{\nu}, X, \theta)$ is the neutrino detection efficiency calculated in the last chapter. 

The next step involves integrating the \textit{acceptance} over the different simulated slant depths(as given in table~\ref{}). This accounts for the \textit{effective mass} target for the neutrino identification over the 6T5 hexagon unit. It is calculated as follows. 

\begin{equation}
  M_{hex}(E_{\nu}) = \int_X A_{hex}(E_{\nu}, X) \cdot dX
\end{equation}

Fig.~\ref{} shows the effective mass for both CC and NC interaction channels for a single 6T5 hexagon unit. Like the detection efficiency it increases with energy. 

Exposure calculation still needs to account for the still needs the detector configuration and its evolution over time. We reduced our array to units of 6T5 hexagons and a full SD array consisting of 1660 stations consists 1420 of these hexagons. Since the establishment of the Pierre Auger Observatory the active number of 6T5 hexagons are monitored every second. This forms a very good indicator for the time evolution of the SD array since any non-working station or large periods of instability are intrinsically recorded in the number of active 6T5 hexagons at that time. The instantaneous number of hexagons, $n_{hex}(t)$ thus can be used as an indicator of detector configurations over time. The $n_{hex}(t)$ were updated and calculated every minute and have an uncertainty of about 1.5\% as mentioned in ~\cite{}. To calculate the energy dependent exposure the effective mass of one 6T5 hexagon is multiplied by the instantaneous number of hexagons and integrated in time. Further, the $\nu$ interaction probability for each flavour(i = $\nu_e, \nu_{\mu}, \nu_{\tau}$) and channel(c = CC,NC) is also folded in. The exposure is given as:

\begin{equation}
  \xi^{i,c}(E_{\nu}) = \frac{\sigma^{i,c}(E_{\nu})}{m_N} \int_{t} M_{hex}^{i,c}(E_{\nu}) \cdot n_{hex}(t) \cdot dt =  \frac{\sigma^{i,c}(E_{\nu})}{m_N} \cdot M_{hex}^{i,c}(E_{\nu}) \cdot N_{hex}
\end{equation}

,$\sigma^{i,c}(E_{\nu})$ is the neutrino nucleon cross-section~\cite{} and $N_{hex}$ is the total number of active 6T5 hexagons integrated in time over the search period. The energy dependent exposure for different flavors and interaction channels is shown in fig~\ref{}. It is necessary to pint out that the double bang showers which can be produced by $\nu_{tau}$ are not taken into account for this analysis. 

An effective or total exposure is calculated by summing all the interaction channels and assuming a 1:1:1 flavour ration at earth(large propagation distances combined with neutrino oscillations) as shown by the red line in fig~\ref{}. 

As seen in the figure the $\nu_e$ CC channels contributes the most to the total neutrino exposure(~85\%). The exposure rises rapidly at lower energies and then mostly flattens with just a slight increase which is due to the energy dependent neutrino cross-section. The shape is due to the neutrino detection efficiency which is small at lower energies as shown in fig~\ref{} in the last section. The next dominant channels are the $\nu_{\tau}$ and $\nu_{\mu}$ which have a similar detection efficiency as the NC channel but have a higher value of cross-section. The NC channel contribution is very small(~5\%) due to the reasons discussed in section ~\ref{}.  

\section{Systematic uncertainties}
\label{sec:det_uncert}



%%% Local Variables:
%%% mode: latex
%%% TeX-master: "mythesis"
%%% End:
